\documentclass[12pt]{article}
\usepackage{graphicx} % Required for inserting images
\usepackage{float}
\usepackage{listings}
\usepackage[dvipsnames]{xcolor}
\usepackage{geometry}
\usepackage[UTF8]{ctex}
\usepackage{amsmath}
\usepackage{amssymb}

\title{Pre-lecture Problems for Lecture 7:\\ Gradient Descent and Newton’s Method}
\author{B10705034 資管三\ 許文鑫}
\geometry{a4paper,scale=0.8}
\begin{document}
\maketitle
\begin{enumerate}
      \item (10 points) Let’s solve
            \begin{equation*}
                  \min_{x\in \mathbb{R}} f(x) = x^2_1 +x^2_2 - x_1x_2
            \end{equation*}
            \begin{enumerate}
                  \item Find the gradient and Hessian of $f (x)$.\\
                        \textbf{Ans.}\\
                        gradient:
                        \begin{equation*}
                              \begin{bmatrix}
                                    2x_1 -x_2 \\
                                    2x_2 - x_1
                              \end{bmatrix}
                        \end{equation*}
                        Hessian:
                        \begin{equation*}
                              \begin{bmatrix}
                                    2  & -1 \\
                                    -1 & 2
                              \end{bmatrix}
                        \end{equation*}
                  \item Let $x^0 = (1, 2)$ be the initial solution. Run one iteration of gradient descent to get the next solution $x^G$. In that iteration, let the step size be that bringing you to the global minimum along the improving direction.\\
                        \textbf{Ans.}
                        \begin{itemize}
                              \item Step 0: $x^0 = (1,2). f(x^0) = 3$
                              \item Step 1:
                                    \begin{itemize}
                                          \item $\nabla f(x^0) = (0,3)$
                                          \item $a_0 = \text{argmax}_{a\ge 0}f(x^0-a\nabla f(x^0)), \text{where} f(x^0-a\nabla f(x^0)) = (1,2-3a)=9a^2-9a+3$
                                                It follows that $a_0 = \frac{1}{2}$
                                          \item $x^1 = x^0 - a_0\nabla f(x^0) = (1,2) - \frac{1}{2}(0,3) = (1, \frac{1}{2})$. Note that $f(x^1) = \frac{3}{4}$\\
                                                $x^G = (1, \frac{1}{2})$
                                    \end{itemize}
                        \end{itemize}
                  \item Let $x^0 = (1, 2)$ be the initial solution. Run one iteration of Newton’s method to get the next solution $x^F$ .\\
                        \textbf{Ans.}
                        \begin{itemize}
                              \item let
                                    \begin{equation*}
                                          f_Q(x) = f(x^0) + \nabla f(x^0)^T(x-x^0) + \frac{1}{2}(x-x^0)^T\nabla^2f(x^0)(x-x^0)
                                    \end{equation*}
                                    be the quadratic approximation of $f(x)$ at $x^0$.
                                    Note that we use the Hessian $\nabla^2f(x^0)$
                              \item We move from $x^0$ to $x^{1}$ by moving to the global minimum of the quadratic approximation:
                                    \begin{equation*}
                                          \nabla f(x^0) + \nabla^2f(x^0)(x^{1}-x^0) = 0,
                                    \end{equation*}
                                    i.e.
                                    \begin{equation*}
                                          x^{1} = x^0 - [\nabla^2f(x^0)]^{-1}\nabla f(x^0)=\begin{bmatrix}
                                                1 \\
                                                2
                                          \end{bmatrix} - \begin{bmatrix}
                                                2  & -1 \\
                                                -1 & 2
                                          \end{bmatrix}^{-1}\begin{bmatrix}
                                                0 \\
                                                3
                                          \end{bmatrix}= \begin{bmatrix}
                                                1 \\
                                                2
                                          \end{bmatrix} - \frac{1}{3}\begin{bmatrix}
                                                2 & 1 \\
                                                1 & 2
                                          \end{bmatrix}\begin{bmatrix}
                                                0 \\
                                                3
                                          \end{bmatrix}
                                    \end{equation*}
                        \end{itemize}
                        $x^F = (0,0)$
            \end{enumerate}
\end{enumerate}
\end{document}