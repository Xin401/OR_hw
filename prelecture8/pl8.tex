\documentclass[12pt]{article}
\usepackage{graphicx} % Required for inserting images
\usepackage{float}
\usepackage{listings}
\usepackage[dvipsnames]{xcolor}
\usepackage{geometry}
\usepackage[UTF8]{ctex}
\usepackage{amsmath}
\usepackage{amssymb}

\title{Pre-lecture Problems for Lecture 8:\\ Linear Programming Duality}
\author{B10705034 資管三\ 許文鑫}
\geometry{a4paper,scale=0.8}
\begin{document}
\maketitle
\begin{enumerate}
      \item (10 points; 2 points each) Consider a primal LP\\
            \textbf{Ans.}
            \begin{align*}
                  \max \quad        & 5x_1 + 3x_2           \\
                  \text{s.t.} \quad & x_1 + x_2 \leq 8      \\
                                    & x_1 +2x_2 \leq 10     \\
                                    & x_1 \geq 0,x_2 \geq 0
            \end{align*}
            \begin{enumerate}
                  \item Find a primal optimal solution $\bar{x}$ by the simplex method.\\
                        \textbf{Ans.}
                        The standard form of the primal LP is
                        \begin{align*}
                              \max \quad       & 5x_1 + 3x_2                \\
                              \text{s.t.}\quad & x_1 + x_2 + x_3 = 8        \\
                                               & x_1 + 2x_2 + x_4 = 10      \\
                                               & x_i \geq 0 \quad i=1,2,3,4
                        \end{align*}
                        \begin{table}[H]
                              \centering
                              \begin{tabular}{cccc|c}
                                    -5 & -3 & 0 & 0 & 0        \\
                                    \hline
                                    1  & 1  & 1 & 0 & $x_3=8$  \\
                                    1  & 2  & 0 & 1 & $x_4=10$ \\
                              \end{tabular}$\rightarrow$
                              \begin{tabular}{cccc|c}
                                    0 & 2 & 5  & 0 & 40      \\
                                    \hline
                                    1 & 1 & 1  & 0 & $x_1=8$ \\
                                    0 & 1 & -1 & 1 & $x_4=2$ \\
                              \end{tabular}
                        \end{table}
                        The optimal solution is $(8,0,0,2)$ with the optimal value 40.
                  \item Formulate the dual LP.\\
                        \textbf{Ans.}
                        \begin{align*}
                              \min \quad        & 8y_1 + 10y_2          \\
                              \text{s.t.} \quad & y_1 + y_2 \geq 5      \\
                                                & y_1 + 2y_2 \geq 3     \\
                                                & y_1 \geq 0,y_2 \geq 0
                        \end{align*}
                  \item Solve the dual LP in any way you like to get a dual optimal solution  $\bar{y}$. Show that $c^T  \bar{x} =  \bar{y}^Tb$, where $c$ and $b$ represent the primal and dual objective coefficients.\\
                        \textbf{Ans.}\\
                        Phase-I:
                        \begin{align*}
                              \min \quad       & y_5+y_6                    \\
                              \text{s.t.}\quad & y_1 + y_2 - y_3+y_5 = 5    \\
                                               & y_1 + 2y_2 - y_4 +y_6= 3   \\
                                               & y_i \geq 0 \quad i=1,2,3,4
                        \end{align*}
                        \begin{table}[H]
                              \centering
                              \begin{tabular}{cccccc|c}
                                    0 & 0 & 0  & 0  & -1 & -1 & 0       \\
                                    \hline
                                    1 & 1 & -1 & 0  & 1  & 0  & $y_5=5$ \\
                                    1 & 2 & 0  & -1 & 0  & 1  & $y_6=3$ \\
                              \end{tabular}$\rightarrow$
                              \begin{tabular}{cccccc|c}
                                    2 & 3 & -1 & -1 & 0 & 0 & 8       \\
                                    \hline
                                    1 & 1 & -1 & 0  & 1 & 0 & $y_5=5$ \\
                                    1 & 2 & 0  & -1 & 0 & 1 & $y_6=3$ \\
                              \end{tabular}\\
                              $\rightarrow$
                              \begin{tabular}{ccccc|c}
                                    0 & -1 & -1 & 1  & 0 & 2       \\
                                    \hline
                                    0 & -1 & -1 & 1  & 1 & $y_5=2$ \\
                                    1 & 2  & 0  & -1 & 0 & $y_1=3$ \\
                              \end{tabular}
                              $\rightarrow$
                              \begin{tabular}{cccc|c}
                                    0 & 0  & 0  & 0 & 0       \\
                                    \hline
                                    0 & -1 & -1 & 1 & $y_4=2$ \\
                                    1 & 1  & -1 & 0 & $y_1=5$
                              \end{tabular}
                        \end{table}
                        Phase-II:
                        \begin{table}[H]
                              \centering
                              \begin{tabular}{cccc|c}
                                    -8 & -10 & 0  & 0 & 0       \\
                                    \hline
                                    0  & -1  & -1 & 1 & $y_4=2$ \\
                                    1  & 1   & -1 & 0 & $y_1=5$
                              \end{tabular}$\rightarrow$
                              \begin{tabular}{cccc|c}
                                    0 & -2 & -8 & 0 & 40        \\
                                    \hline
                                    0 & -1 & -1 & 1 & $y_4 = 2$ \\
                                    1 & 1  & -1 & 0 & $y_1 = 5$
                              \end{tabular}\\
                              $\rightarrow$
                              \begin{tabular}{cccc|c}
                                    0 & -2 & -8 & 0 & 40        \\
                                    \hline
                                    0 & -1 & -1 & 1 & $y_4 = 2$ \\
                                    1 & 1  & -1 & 0 & $y_1 = 5$
                              \end{tabular}
                        \end{table}
                        We get $\bar{y} = \begin{bmatrix}
                                    5 \\ 0
                              \end{bmatrix}$, and the optimal is value 40.
                        \begin{align*}
                              c^T\bar{x} = \begin{bmatrix}
                                                 5 & 3
                                           \end{bmatrix}\begin{bmatrix} 8 \\0
                                                        \end{bmatrix} = 40 = \bar{y}^Tb\begin{bmatrix}
                                                                                             5 & 0
                                                                                       \end{bmatrix}\begin{bmatrix}
                                                                                                          8 \\10
                                                                                                    \end{bmatrix}
                        \end{align*}
                  \item Use the primal optimal basis $B$ you found in Part (a) to verify that $c^T_BA^{-1}_B = \bar{y}^{T}$.\\
                        \textbf{Ans.}
                        \begin{align*}
                              x_B = \begin{bmatrix}
                                          x_1 \\x_4
                                    \end{bmatrix},c_B = \begin{bmatrix}
                                                              5 \\ 0
                                                        \end{bmatrix},
                              A_B = \begin{bmatrix}
                                          1 & 0 \\1 & 1
                                    \end{bmatrix},A_B^{-1} = \begin{bmatrix}
                                                                   1 & 0 \\-1 & 1
                                                             \end{bmatrix}
                        \end{align*}
                        then we can verify that
                        \begin{align*}
                              c_B^TA_B^{-1} = \begin{bmatrix}
                                                    5 & 0
                                              \end{bmatrix}\begin{bmatrix}
                                                                 1 & 0 \\-1 & 1
                                                           \end{bmatrix} = \begin{bmatrix}
                                                                                 5 & 0
                                                                           \end{bmatrix} = \bar{y}^T
                        \end{align*}
                  \item Find the shadow prices for the two primal constraints.\\
                        \textbf{Ans.}\\
                        Shadow prices are the dual optimal solution, so
                        \begin{align*}
                              x_1 + x_2 \leq 8 \rightarrow \text{shadow price} = 5 \\
                              x_1 + 2x_2 \leq 10 \rightarrow \text{shadow price} = 0
                        \end{align*}
            \end{enumerate}
\end{enumerate}
\end{document}