\documentclass[12pt]{article}
\usepackage{graphicx} % Required for inserting images
\usepackage{float}
\usepackage{listings}
\usepackage[dvipsnames]{xcolor}
\usepackage{geometry}
\usepackage[UTF8]{ctex}
\usepackage{amsmath}
\usepackage{amsfonts}


\title{Pre-lecture Problems for Lecture 4:\\ Integer Programming}
\author{B10705034 資管三\ 許文鑫}
\geometry{a4paper,scale=0.8}
\begin{document}
\maketitle
\begin{enumerate}
    \item[3.](10 points) A firm is considering importing some products to sell in a local market. The unit price and cost for product $i$ are $p_i$ and $C_i$, respectively. While the prices may be determined by the firm, the costs are given and fixed. For product $i$, the demand volume is $A_i - B_ip_i$, where $A_i$ and $B_i$ are all given. The fixed cost of importing product $i$ is $K_i$. In other words, the cost $K_i$ is incurred if and only if a positive amount of product $i$ is imported. Note that given the existence of the fixed cost, to maximize the firm’s profit it may be good to import only a subset of products. Formulate a mathematical program that can find a purchasing plan that maximizes the firm's profit. Determine whether your program is a linear program, a linear integer program, a nonlinear program, or a nonlinear integer program.\\
    \textbf{Ans. }\\
    The profit of importing product $i$ is $(p_i-C_i)(A_i - B_ip_i) - K_i$. Let $z_i$ be the binary variable that indicates whether the firm imports product $i$. $I$ is the set of products. The mathematical program is:
    \begin{align*}
        \max \quad & \sum_{i\in I}(p_i - C_i)(A_i - B_ip_i)z_i - K_iz_i \\
        \text{s.t.} \quad
                   & A_i - B_ip_i \geq 0 \quad \forall i\in I           \\
                   & p_i \geq C_i \quad \forall i\in I                  \\
                   & z_i \in \{0,1\} \quad \forall i\in I
    \end{align*}
    This is a nonlinear integer program.
\end{enumerate}
\end{document}

