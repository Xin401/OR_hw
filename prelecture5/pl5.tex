\documentclass[12pt]{article}
\usepackage{graphicx} % Required for inserting images
\usepackage{float}
\usepackage{listings}
\usepackage[dvipsnames]{xcolor}
\usepackage{geometry}
\usepackage[UTF8]{ctex}
\usepackage{amsmath}


\title{Pre-lecture Problems for Lecture 5:\\
The Simplex Method}
\author{B10705034 Wen-Xin, Xu}
\geometry{a4paper,scale=0.8}
\begin{document}
\maketitle
\begin{enumerate}
    \item[3.] (10 points) Consider the following LP
        \begin{align*}
            \text{max }  & x_1 + 2x_2 + x_3                       \\
            \text{s.t. } & x_1 + x_2 \leq 10                      \\
                         & x_2 + x_3 \leq 8                       \\
                         & x_i \geq 0 \quad \forall i = 1, ...,3.
        \end{align*}
        \begin{enumerate}
            \item (5 points) Find all the basic solutions and basic feasible solutions for the LP.\\
                  \textbf{Ans. }\\
                  We can first change the LP into standard form:
                  \begin{align*}
                      \text{max }  & x_1 + 2x_2 + x_3                           \\
                      \text{s.t. } & x_1 + x_2  \quad\quad +x_4 \quad\quad = 10 \\
                                   & \quad \quad x_2 + x_3 \quad \quad +x_5 = 8 \\
                                   & x_i \geq 0 \quad \forall i = 1, ...,5.
                  \end{align*}
                  Then we can find all the basic solutions and basic feasible solutions for the LP.
                  \begin{table}[H]
                      \centering
                      \begin{tabular}{|c|c|c|c|c|c|c|}
                          \hline
                          Basis       & feasible & $x_1$ & $x_2$ & $x_3$ & $x_4$ & $x_5$ \\
                          \hline
                          $(x_1,x_2)$ & Yes      & 2     & 8     & 0     & 0     & 0     \\
                          $(x_1,x_3)$ & Yes      & 10    & 0     & 8     & 0     & 0     \\
                          $(x_1,x_5)$ & Yes      & 10    & 0     & 0     & 0     & 8     \\
                          $(x_2,x_3)$ & No       & 0     & 10    & -2    & 0     & 0     \\
                          $(x_2,x_4)$ & Yes      & 0     & 8     & 0     & 2     & 0     \\
                          $(x_2,x_5)$ & No       & 0     & 10    & 0     & 0     & -2    \\
                          $(x_3,x_4)$ & Yes      & 0     & 0     & 8     & 10    & 0     \\
                          $(x_4,x_5)$ & Yes      & 0     & 0     & 0     & 10    & 8     \\
                          \hline
                      \end{tabular}
                      \caption{All the basic solutions and basic feasible solutions for the LP.}
                  \end{table}
            \item (5 points) Use the simplex method to solve that LP. In the first iteration, enter x1. Write down all the iterations, an optimal solution, and the associated objective value.\\
                  \textbf{Ans. }\\
                  Let $z = x_1 + 2x_2 + x_3$, we can write the LP in the following form:
                  \begin{align*}
                      z- x_1 - 2x_2 - x_3 \quad\quad\quad\quad =0           \\
                      \quad\quad x_1 + x_2  \quad\quad +x_4 \quad\quad = 10 \\
                      \quad\quad\quad \quad x_2 + x_3 \quad \quad +x_5 = 8  \\
                  \end{align*}
                  Then we can write the initial tableau:
                  \begin{table}[H]
                      \centering
                      \begin{tabular}{ccccc|c}
                          -1 & -2 & -1 & 0 & 0 & 0          \\
                          \hline
                          1  & 1  & 0  & 1 & 0 & $x_4 = 10$ \\
                          0  & 1  & 1  & 0 & 1 & $x_5 = 8$  \\
                      \end{tabular}
                      \caption{Initial Tableau}
                  \end{table}
                  We can see that the coefficient of $x_1$ in the objective row is -1, so we can enter $x_1$ into the basis. Then we can write the next tableau:
                  \begin{table}[H]
                      \centering
                      \begin{tabular}{ccccc|c}
                          0 & -1 & -1 & 1 & 0 & 10         \\
                          \hline
                          1 & 1  & 0  & 1 & 0 & $x_1 = 10$ \\
                          0 & 1  & 1  & 0 & 1 & $x_5 = 8$  \\
                      \end{tabular}
                      \caption{Second Tableau}
                  \end{table}
                  We can see that the coefficient of $x_2$ in the objective row is -1, so we can enter $x_2$ into the basis. Then we can write the next tableau:
                  \begin{table}[H]
                      \centering
                      \begin{tabular}{ccccc|c}
                          0 & 0 & 0  & 1 & 1  & 18      \\
                          \hline
                          1 & 0 & -1 & 1 & -1 & $x_1=2$ \\
                          0 & 1 & 1  & 0 & 1  & $x_2=8$ \\
                      \end{tabular}
                      \caption{Third Tableau}
                  \end{table}
                  We can see that all the coefficients in the objective row are non-negative, so the current solution is optimal. The optimal solution is $x_1=2$, $x_2=8$, $x_3=0$, and the associated objective value is $z=2+2*8+0=18$.
        \end{enumerate}
\end{enumerate}
\end{document}

